\section{Conclusion and Future Work}
Blockchain is a new and innovative approach for tracking, tracing and auditing data. In this thesis we developed a decentralized system for tracking shipments. This system was implemented as a proof of concept. An enterprise level version of this solution will need additional supporting technologies and systems to operate smoothly. Some of these improvements are discussed in \ref{improve}. Our system can reduce cost and increase efficiency and transparency. This is achieved by increasing automation during the supply chain cycle. Traditional systems rely on employees manually entering shipping and logistics data in the monitoring systems. This is inefficient and prone to human errors. In our proposed system critical data is communicated automatically using IoT Nodes. Efficiency is further increased by eliminating human errors and providing real time access to critical data for all relevant stake holders.

Managers and supply chain engineers can subscribe to important events and get real time information. This allows them to quickly react to changing situations. Our System democratizes trust by giving every stake holder access to the same information. Shippers, suppliers and company executives all have access to the same data. This system is fully decentralized and is not under the control of any party. Our system can be easily configured or extended to suit the needs of different industries. It can be used for quality control of grocery and other consumer items. It can enable end users or customers to have complete confidence that the product was stored, shipped and handled in accordance with strict safety and regulatory standards. It can be deployed to facilitate frictionless trade by reducing delays caused by customs and border checks. Custom procedures could be automated with the IoT Node securely communicating customs declarations at the border. Similarly, quality and compliance checks can be expedited. The IoT package would send compliance information to the concerned parties in advance or on demand.

\subsection{Limitations and Future Work} \label{improve} 
The current design of the proposed decentralized system is a simplified version to test the potential of using blockchain for Supply Chain Management and Shipment Tracking. Due to the complexities of Ethereum and the current state of blockchain technology many assumptions and tradeoffs were established for this project. As blockchain technology and platform matures certain features of this prototype can be improved. Some areas of improvement are identified below:

\textbf{Flexibility:} The current design requires the IoT Node to have access to a reliable internet connection throughout the shipping cycle. This insures shipping violations and logistics data is communicated in real time with all relevant stake holders. In real scenarios guaranteeing access to reliable internet connection may not be possible. The IoT design can be made more flexible by insuring that it functions in cases even when internet is not available. If the node senses loss of connectivity, it can start logging data. The logged data would be communicated with the blockchain and IPFS whenever internet is available. If human shippers are making deliveries, the IoT Node can use Bluetooth to communicate violations with the shipper in real time. 

\textbf{Private Blockchain:} Private or consortium blockchain platforms can be explored as an alternative for deploying this system. Private blockchains have obvious advantages in terms of security and performance compared to public blockchains. The obvious disadvantage is the lack of real decentralization and the need to trust third parties. 

\textbf{Integrate State Channel based Solutions:} Off-chain solutions like Raiden and Perun have many benefits. They increase performance, improve reliability and efficiency and help to reduce costs. The current solution uses Raiden Network for making payments only. This is due to design limitations of the payment channel technology on which Raiden is based on. Perun is an alternative off-chain solution for Ethereum blockchain. It proposes using state channels which are capable of running any type of Smart Contract. If Perun lives up to its promise, it can be an interesting way to resolve some of the risks associated with running this system on a public blockchain. It would make our system resilient against network congestions. It can reduce operational costs and increase performance as channel transactions are free and instantaneous. 

\textbf{Miscellaneous Design Improvements:} 
Certain design improvements can be made to the existing system. The current system relies on suppliers or shippers to manually initialize the IoT Node with the tracking number. This step can be easily automated by assigning static device ID’s to each sensor nodes. A remote server can be used to assign dynamic tracking number to device ID’s at the start of each shipping cycle. To improve data confidentiality log files can be encrypted before uploading to IPFS. Another design improvement worth considering is integrating a supplier rating system in our system. The suppliers or shippers would be given a bad rating for less severe violations. Each violation is assigned a severity level: medium, low, high and critical. The organization can use these ratings to keep track of the quality of service provided by their partners (shippers and suppliers). 
