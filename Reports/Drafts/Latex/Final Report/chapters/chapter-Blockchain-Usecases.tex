\section{Blockchain Applications}
Companies and People across the globe are exploring applications of blockchain technology across several industries. This chapter details blockchain applications and projects from various sectors including finance, Internet of Things, Supply chain Management and File hosting and sharing. 
\subsection{Crypto Currencies}
Bitcoin was the first ever application of blockchain technology. It was proposed and developed shortly after the financial crash of 2007. This crash was caused by the irregularities in the existing centralized financial institutions and banks. The biggest problem with traditional banking is that it centralizes trust in a few large financial institutions and banks. This system works only as long as these banks operate reliably and responsibly. The financial crash of 2007 showed that they cannot be always trusted to act responsibly \cite{misc:009}. Bitcoin was developed to address problems prevalent in existing centralized financial system. It solves these problems by developing a system which removes all central entities and trust is established through a system of checks and balances [\ref{Blockchain}]. Cryptocurrency sector experienced huge growth in terms of users and investment during 2016 and 2017 and at one point was worth close to 500 billion dollars \cite{misc:010}. Unfortunately, most of this growth was due to speculative investments and has not yet translated into large scale adoption of Cryptocurrencies in everyday business and finance. Blockchain powered financial systems have huge promise provided they are able to solve some of the challenges outlined in [\ref{scaling}].
\subsection{Internet of Things}
IoT is the next wave of automation promising to disrupt industrial and domestic structures and processes. The billions of smart devices coming online could transform homes, cities, offices and factory floors \cite{misc:008}. \textit{“IoT holds the promise to expand business processes and to accelerate growth. However, the rapid evolution of the IoT market has caused an explosion in the number and variety of IoT solutions, which created real challenges as the industry evolves, mainly, the urgent need for a secure IoT model to perform common tasks such as sensing, processing, storage, and communicating”} \cite{misc:002}. Currently IoT ecosystems are realized using brokered communication models based on client/server paradigm. Devices are connected through cloud servers using the internet even if they are few feet away from each other. Further more centralized models struggle to scale up to meet the demands of billions of users or devices \cite{misc:004}. Blockchains offer an intriguing alternative as a secure and decentralized IoT command, control and communication model. Blockchain and Smart contract based solutions should be more manageable and scalable then traditional ones. Blockchains are inherently tamper resistant hence they will prevent one or more rogue devices from causing a complete system breakdown across a home, factory, or transportation system. Blockchain in IoT sphere has the potential to revolutionize several industries and businesses \cite{misc:008} \cite{misc:004} \cite{misc:002}.
 
There are several exciting projects working towards this goal. Sections [\ref{Adept}]and [\ref{Filament}] describe two projects at the forefront of blockchain and IoT.


\textbf{Smart Washing Machine}
This example is a realization of proof of concept proposal from Samsung. \textit{“Imagine a washer that autonomously contacts suppliers and places orders when it’s low on detergent, performs self-service and maintenance, downloads new washing programs from outside sources, schedules its cycles to take advantage of electricity prices and negotiates with peer devices to optimize its environment”} \cite{misc:004}. If the machine is connected to some sort of ledger be it private or public, it can automatically pay the detergent suppliers and repairmen \cite{misc:004} \cite{misc:005}.

\subsubsection{Autonomous Decentralized Peer-to-Peer Telemetry (ADEPT)} \label{Adept}
It is a joint venture between IBM and Samsung Electronics. It is developed to serve as a validation platform for projects proposing to connect IoT and Blockchain. It envisions network of devices that are capable of autonomously maintaining themselves. Adept is working towards integrating IBMs Watson IoT platform with blockchain technologies. It is currently still in development stage but proof of concept has already been implemented \cite{misc:005}. It uses blockchains as the backbone of the system using a mix of proof of work and proof stake to secure transactions. The ADEPT architecture supports three foundational functions.
\begin{itemize}
  \item Peer to Peer encrypted messaging using a secure messaging protocol called TELEHASh \cite{misc:005}.
  \item Decentralized file sharing based on BitTorrent protocol \cite{misc:005}.
\item Decentralize device coordination and control
In the absence of centralize controller, device control and coordination becomes significantly challenging. Adept uses Ethereum or hyper ledger blockchain to implement this in a trustless secure fashion \cite{misc:005}.

\end{itemize}
Ultimately it will enable IoT devices to send data to private blockchain ledgers for inclusion in shared transactions with tamper-resistant records. Devices will be able to communicate with the blockchain in order to update or validate smart contracts. This will improve transparency and reduce conflicts by empowering all stake holders. Each stakeholder will have access to the same data and could easily verify all transactions \cite{misc:005} \cite{misc:004}. 
\clearpage


\subsubsection{Filament} \label{Filament}
Filament is a technology stack that \textit{“enables devices to discover, communicate, and interact with each other in a fully autonomous and distributed manner”} \cite{misc:006}. Its main focus is industrial Internet of Things. \textit{“The Filament technology stack is built upon five key principles: Security, Privacy, Autonomy, Decentralization, and Exchange (SPADE)”} \cite{misc:006}. Filament uses Telehash for secure encrypted device to device communication. Secure Identity is provided by blockchain. Once a secure communication channel has been established between devices, smart contracts are used to interact with them, or to enable them to transact with each other. Smart Contracts in Filament run directly on device and accept or run transactions from other devices based on contractual terms. It uses a protocol suite called “JOSE” (Javascript Object signing and Encryption) to implement smart contracts on the devices \cite{misc:007}. In order to enable micro transactions on these embedded devices authors of the Filament project propose a solution called Penny Bank. It allows the devices to exchange small amounts of value with each other without involving the blockchain for every single transaction thereby avoiding heavy transaction fees \cite{misc:006}.

\subsection{Supply Chain Management}
Blockchains allow secure and permanent documentation of transactions in a decentralized ledger. They can be monitored transparently by all parties. This can improve efficiency and reduce human mistakes and time delays. It can also enable users to verify authenticity of products by tracking them from their origin. An example of this is securing supply chains of diamonds from mine to consumers. IBM’s Hyperledger Fabric is one of the proposals working in this field. It is permissioned blockchain infrastructure designed specifically for handling supply chain management tasks. Using blockchain technologies customers can verify when and where a diamond was mined, all the places it passed through during its journey to the retailer, and whether or not during any step of the supply chain it crossed any moral or legal grey areas i.e. if it’s a blood diamond.
\subsubsection{Skuchain}
Sku chain is a platform that uses blockchain to provide security, efficiency and transparency to supply chains. Today’s supply chain management tools such as ERP systems, Inventory Management systems, Letters of credits, Purchase and Invoicing tools have significant friction and problems interfacing with each other. This results in increased costs and delays in every process of the supply chain. SKuchain proposes tools in order to resolve some of these issues \cite{misc:015}.

\textbf{IMT:}
\textit{“IMT provides inventory financing that de-risks transactions and unlocks capital opportunities for the entire supply chain. The original contract between the buyer and seller is assigned to IMT in the blockchain. This acts as a Blockchain Based Security Interest that provides the collateral to an investor in the IMT fund. IMT uses its funds to purchase goods from the seller and stores them at a VMI warehouse. Finished goods are shipped pursuant to a purchase order from the buyer, a process covered by insurance. The buyer then pays IMT for the goods”} \cite{misc:015}.

\textbf{Brackets}
Smart contracts on the skuchain are called brackets. They are cryptographically secured. They provide some key advantages. 

\begin{itemize}

\item They are designed to release collateral as a result of being triggered automatically by real world events \cite{misc:015}. 
\item  They improve transparency for all participants by providing real time view of transaction state \cite{misc:015}. 
\item  \textit{“It enhances liquidity of collateralized assets in a supply chain by improving upon current trade finance instruments such as Factoring, PO Financing and Vendor Managed Inventory Financing. It also creates the opportunity for Deep Tier Financing”} \cite{misc:015}. 
\end{itemize}

\textbf{PopCodes}
\textit{“Popcodes are Proof of Provenance codes, a crypto-serialization solution to track flow of goods on SKU level. They provide bank-grade traceability to track physical value in the supply chain. Popcodes are sophisticated in their ability to track sub-assemblies, parts and raw materials used to make a finished product. Using Popcodes, an enterprise can gain JIT visibility across the entire supply chain ecosystem, enabling optimal agility and planning. It also provides end-consumer visibility into the entire history of the product”} \cite{misc:015}.

\subsubsection{Provenance}
This project is working on using blockchain technology to enable secure traceability of certifications and other salient information in the supply chain. It aims to become a platform for verifying authenticity of goods. \textit{“Provenance enables every physical product to come with a digital ‘passport’ that proves authenticity (Is this product what it claims to be?) and origin (Where does this product come from?), creating an auditable record of the journey behind all physical products”} \cite{paper:006}. They are creating a decentralized app for solving certification and chain of custody challenge in sustainable supply chains. It proposes a system to assign and verify certain properties of physical products using the blockchain. There are six different actors involved in the proposed scheme namely.
\begin{itemize}
\item Producers 
\item Manufacturers
\item Registrars, they provide unique identity to other actors in the scheme 
\item Standards organizations, which define the rules of a certain scheme (e.g., Fairtrade) 
\item Certifiers and auditors 
\item Customers
 
\end{itemize}
The architecture in their white paper \cite{paper:006} consists of number of modular programs. Namely Registration program, Standards programs, Production programs, and Manufacturing programs. Each contract is deployed on the blockchain independently but since all of them work within the same blockchain system they can interact without friction. Technologies such as NFC, RFID, barcodes, and digital tags link physical products to their digital representation on the blockchain. Furthermore, user facing application in the form of smart phone applications will facilitate access to the blockchain. They will aggregate and display information to customers in real time, detailing every step of the supply chain \cite{paper:006}.

\subsection{Filesharing}
Filecoin, SiaCoin and Storj are some of the proposals for creating a decentralized platform for filesharing, storage and cloud computing using the blockchain. The idea is simple users instead of uploading files to a central cloud server hosted at google, Microsoft or Dropbox files are shredded, encrypted and spread across the distributed file storage network based on the blockchain. Only the uploader holds the keys to call smart contracts to decrypt and reassemble the files. People participating as hosts in the network rent out their storage spaces and get paid in return for the services they provide \cite{misc:016}.
\subsubsection{FileCoin}
\textbf{Filecoin}
\textit{ “Filecoin is a decentralized storage network that is auditable and publically verifiable. Clients pay miners for data storage and retrieval. Clients offer data storage and disk space in exchange for payments. The network achieves robustness by replicating and dispersing content while automatically detecting and repairing replica failures”} \cite{paper:007}.

\textbf{Proof of Storage}
Proof of storage is a class of decentralized challenge response protocols. They allow a storage provider or host to efficiently verify the integrity of the data stored on their device to their users or clients. The client sends encrypted version of its data to the hosting node for storage, while keeping a small portion of that data himself so he can cryptographically verify the integrity of data stored on the hosting node at any time \cite{paper:007}.

\textbf{Proof of Replication}
\textbf{Proof of Replication}
Filecoin introduced a special form of proof of storage protocols called Proof of Replication. It is an extension of Proof of Storage protocol. It enables a miner to convince a user that some data D has been successfully replicated to its own unique physical storage S. It uses challenge/ response protocol to achieve this \cite{paper:007}. Traditional PoS protocols are limited in that they only prove that a miner or host was in possession of data at the time of challenge/response. Filecoin developed PoR protocol in order to provide stronger guarantees against Sybil attacks, Outsourcing attacks and Generation attacks \cite{paper:007}. These attacks are exploited by malicious nodes to gain reward for storage that they are not actually providing. Such greedy miners reduce the overall capacity and performance of the network. These attacks are discussed below.

\textbf{Sybil Attacks}
A malicious attacker may want to claim the reward for storing multiple copies of some data D. They can cheat the system in traditional PoS protocols by claiming to store multiple copies using Sybil identities, while in reality only storing one physical copy of the D \cite{paper:007}.

\textbf{Outsourcing Attacks}
A malicious miner could exploit the system by quickly fetching the data D he is claiming to store from other nodes at the time of challenge/response \cite{paper:007}.

\textbf{Generation Attacks}
A malicious miner could rely on small but efficient program to quickly generate the data D when it is requested. This could allow such an attacker to gain reward for storing large amounts of data even when he physically does not have the capacity to do so \cite{paper:007}.

\textbf{Proof of Space Time}
Proof of Space Time is a novel implementation of PoS. It allows a user to verify that the data was being stored by the miners throughout a period of time. A natural way to verify this would be by repeatedly sending challenges to the miners. This implementation would quickly bottle neck the network by flooding it with larger number of communication requests. Proof of space time instead requires a storage provider to produce a sequential proof of storage for a period of time and then recursively compose them together to generate a complete proof \cite{paper:007}.





%\subsection{Blockchain based Smart Grid}
