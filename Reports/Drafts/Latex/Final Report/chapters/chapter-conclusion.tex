\section{Conclusion and Future Work}
\ points to elabortae
Make the design flexible i.e. so that it does not require always on internet connection.
i.e. it communicates shipping and logistical data whenever internet or network connection is available.
Use Perun to get the entire smart contract offchain to reduce dependency on on chain and resiliance against network congestions.
etct etch
\\ log files uploaded to IPFS can be encrypted with either public key or secret key techniques to provide complete data security 
further try and solve the confidentiality issue.?
To conclude\ldots
Conclusion

In this thesis we developed a proof of concept system , This demonstrated the usefulness of such a system. A corporate or enterprise level solution will need additional supporting technologies and systems for it function smoothly. One area of improvement would be integrating RF chips in the IoT nodes that shippers Scan using their phones or RF readers to send change in shipper notifications etc. We could develop a smart phone application which could be used to scan packages and get all relevant information using NFC or Bluetooh. Managers and Supply chain engineers can use this application to subscribe to important events such as change in shipper or custom checks etc. to get real time information every where. This could help them plan better supply chain processes in the future.


Our system can reduce cost, increase efficency, and increase transparency. It reduces cost by increasing automation, data is sent autmatically from IoT nodes to the blocckhain,   The company and shippers does not need to hire dedicated persons for entering data in suplly chain monitoring systems which is the case in traditional systems. Shippers and Company reps can subscribe to important events.

Efficency is increased, by eliminated human errors, and providing real time critical data to the relevant stake holders. 

This system establishes trust between parties by giving every stake holder access to the same information. Shippers, suppliers and company execs all have access to the same data, and the system is not under the control of any one party.



\subsection{Benefits}
%Benefits
I.e allows complete transparency to all participants of the supply chain. Enables / Gives end users or customers to have complete confidence that the product was stored, shipped and handled in accordance to strict safety standards and regulations. The transparency brought by this solution to  the supply chain life cycle makes the job of government regulators and safety inspectors much simpler.

Although our test case calls for only monitoring environmental conditions and location of a shipment or package, This technology can be easily extended to monitor other critical events like custom checks or quality checks etc. This has the potential to reduce delays at borders, custom procedures could be automated with IoT node securely communicating custom declarations to the systems at the border. Similarly quality and compliance checks can be expedited, IoT package would send quality and comliance information to the concerned parties in advance or on demand. We can greatly reduce the number of humans involved in these processes by automating
processes like custom declarations and compliance certifications.
   
Improvements 
before shipping starts the raspberry pi must be configured with the correct tracking number. The supplier must do this manually, how ever this can be automates easily by giving a static device number to each sensor node and then assigning new tracking number to a device each time raspbery pi / iot node starts. Wehn an IoT node starts it will query the remote server with its deviceID to get a tracking number for the new shipping cycle. 

We can integrate a supplier rating system in our dapp to keep track of ratings. The supplier and shipper would be given a bad rating for less svere violations. Each violation carries a severity level i.e. medium to low would result in a bad rating. Organization can use the ratings to keep track of the quality of service provided by the shipper or supplier. 
Mention another benefit of using a private blockchain would be to 
