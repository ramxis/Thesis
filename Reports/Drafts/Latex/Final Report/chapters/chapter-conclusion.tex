\section{Conclusion and Future Work}
\ points to elabortae
Make the design flexible i.e. so that it does not require always on internet connection.
i.e. it communicates shipping and logistical data whenever internet or network connection is available.
Use Perun to get the entire smart contract offchain to reduce dependency on on chain and resiliance against network congestions.
etct etch
\\ log files uploaded to IPFS can be encrypted with either public key or secret key techniques to provide complete data security 
To conclude\ldots

Improvements 
before shipping starts the raspberry pi must be configured with the correct tracking number. The supplier must do this manually, how ever this can be automates easily by giving a static device number to each sensor node and then assigning new tracking number to a device each time raspbery pi / iot node starts. Wehn an IoT node starts it will query the remote server with its deviceID to get a tracking number for the new shipping cycle. 

\subsection{Benefits}
%Benefits
I.e allows complete transparency to all participants of the supply chain. Enables / Gives end users or customers to have complete confidence that the product was stored, shipped and handled in accordance to strict safety standards and regulations. The transparency brought by this solution to  the supply chain life cycle makes the job of government regulators and safety inspectors much simpler.