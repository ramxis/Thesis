\documentclass[parskip,accentcolor=tud9c]{tudreport}

%\usepackage{tustyle}
\usepackage[ngerman]{babel}
\usepackage[utf8]{inputenc}
%\usepackage{tudfonts}
\usepackage{siunitx}
\usepackage{booktabs}
\usepackage{listings}
\usepackage[markup=nocolor]{changes}%Es gibt die Möglicheit, Text durchzustreichen
\usepackage[activate={true,nocompatibility},final,tracking=true,kerning=true,spacing=true,factor=1100,stretch=10,shrink=10]{microtype}
\usepackage{url}
\usepackage{blindtext}
\usepackage{setspace}

\usepackage[
nonumberlist, 			% keine Anzeige von Seitenzahlen der Einträge im Abkürzungs-/Symbolverzeichnis
nogroupskip, 			% keine Gruppierung der Einträge nach Anfangsbuchstaben & keine zusätzlicher vertikaler Abstand bei Änderung des Anfangsbuchstabens
acronym, 			% ein Abkürzungsverzeichnis erstellen
toc, 					% Einträge im Inhaltsverzeichnis für das Abkürzungs- und Symbolverzeichnis
translate=babel 		% Übersetzung von Überschriften, etc. ins Deutsche
]{glossaries}			% Paket zur Erstellung von Abkürzungs-, Symbolverzeichnissen, etc.

%%Für MiKTeX-Nutzer (Biber):
%\usepackage[backend=biber, style=numeric-comp, bibstyle=numeric, citestyle=numeric]{biblatex}
%Für TeXLive-Nutzer (Bibtex):
\usepackage[backend=bibtexu, style=numeric-comp, bibstyle=numeric, citestyle=numeric]{biblatex}
%Literaturverzeichnis wird gemäß den gewünschten Optionen gestaltet

\usepackage{hyperref}
%setzt Referenzen so, dass durch Klicken an die entsprechende Stelle gesprungen werden kann

% % % % % % % % % % % % % % % % % % % 
%										 %
%	Festlegen von Einstellungen der Pakete		%
%										 %
% % % % % % % % % % % % % % % % % % % 

\geometry{top=2.5cm, left=2.5cm, right=2.5cm, bottom=2cm}			
% Definition der Seitenränder
\graphicspath{{img/}}															% legt Unterverzeichnisse fest in denen \includegraphics standardmäßig nach Bilddateien sucht
\sisetup{
	exponent-product = \cdot,		% setzt einen Malpunkt als Trennzeichen bei der Verwendung von Exponenten
	output-decimal-marker = {,},		% setzt Komma als Dezimalzeichen (im Englischen ist es ein Punkt)
	per-mode = symbol,			% legt fest, dass Einheiten im Nenner mit einem Bruchstrich statt mit negativen Exponenten dargestellt werden
	bracket-unit-denominator = false,	% mehrere Einheiten im Nenner werden nicht durch eine Klammer umschlossen
	range-phrase = ~--~,			% bei Verwendung von \SIrange wird ein Bindestrich zwischen den beiden Zahlen verwendet
	number-unit-product = \text{~},	% legt den Abstand zwischen Zahl und Einheit fest
	detect-all,					% hiermit werden Veränderungen der Schrift auch bei der Darstellung der Zahlen und Einheiten berücksichtigt
	list-final-separator = ~und~,		% Übersetzung
	list-pair-separator = ~und~,		% Übersetzung
}
%%Sorgt für die farbigen Unterlegungen von Internetlinks und internen Verweisen.
%\hypersetup{ %
%	colorlinks = false,		% true: stellt Verknüpfungen wie Links, Zitate und Verweise farbig dar, false: rahmt Links, Zitate und Verweise in einen farbigen Rahmen ein
%	linkcolor = red,
%	urlcolor = cyan,		% Schriftfarbe für URLs
%	citecolor = green,		% Schriftfarbe für Zitate
%	pdfborder = 1 0 1,		% legt Rahmen fest, sofern verwendet
%	linkbordercolor = red,	% Rahmenfarbe bei Verweisen (white = transparent bei weißem Hintergrund)
%	urlbordercolor = cyan,		% Rahmenfarbe bei URLs
%	citebordercolor = green,		% Rahmenfarbe bei Zitaten (white = transparent bei weißem Hintergrund)
%	bookmarksopen = false,		% legt fest ob die Lesezeichenleiste beim Öffnen des PDFs expandiert ist oder nicht
%	bookmarksnumbered = false,	% legt fest ob die Kapitelnummern in den Lesezeichenbaum übernommen werden
%}

% % % % % % % % % % % % % % % % % % %
%										 %
% 	Anpassung des Literaturverzeichnisses		%
%										 %
% % % % % % % % % % % % % % % % % % % 

\addbibresource{bibliography.bib}		
% läd die Literaturdatenbank aus dem aktuellen Arbeitsverzeichnis
\ExecuteBibliographyOptions{%
	url = false,
	%dashed=false			%in Bibliographie-Stilen, welche das Literaturverzeichnis nach Autorennamen sortieren werden bei setzen der Einstellung auf true bei mehreren Quellen eines Autors die Einträge untereinander gruppiert und der Autorenname durch einen Strich ersetzt
	bibencoding=utf8, % wenn .bib in utf8, sonst ascii
	bibwarn=true, % Warnung bei fehlerhafter bib-Datei
	sorting=none % gibt Einträge im Literaturverzeichnis in der Reihenfolge aus, in der sie zitiert wurden	
}%

\DefineBibliographyStrings{ngerman}{%
	bibliography={Literaturverzeichnis},			% setzt die Überschrift des Literaturverzeichnis
	urlseen          = {Zugriff\addcolon}				% ändert die Beschriftung des Datums bei URLs von "besucht am" auf "Zugriff:"
}

\DeclareNameAlias{default}{last-first}  			
%Im Literaturverzeichnis folgt die Darstellung der Autorennamen der Darstellung Nachname, Vorname (abhängig vom Stil, muss dies angepasst werden um die Effekt zu erhalten, siehe: http://projekte.dante.de/DanteFAQ/BiblatexReihenfolgeAutoren)
\renewcommand*{\mkbibnamelast}[1]{\textsc{#1}}		
%Setzen des Nachnamens des Autors in Kapitälchenschrift


% % % % % % % % % %% % % % % % % % % % % % % % % 
%												              %
%	Neu-Definitionen und Änderungen bestehender Befehle		%
%													      %
% % % % % % % % % % % % % % % % % % % % % % % % % 

% Neudefinition der Itemize-Umgebung um die Abstände zwischen einzelnen Stichpunkten von Aufzählungen zu Verringern
	\let\olditemize\itemize
	\renewcommand{\itemize}{
		\olditemize
		\itemsep4pt
	}

% Neudefinition der Enumerate-Umgebung um die Abstände zwischen einzelnen Stichpunkten von Aufzählungen zu Verringern
	\let\oldenumerate\enumerate
	\renewcommand{\enumerate}{
		\oldenumerate
		\itemsep4pt
	}

\addto\extrasngerman{\def\figureautorefname{Abb.}}							
% legt fest, dass bei Verweis auf eine Abbildung im Text, das Wort Abbildung mit Abb. abgekürzt wird
\addto\extrasngerman{\def\tableautorefname{Tab.}}							
% legt fest, dass bei Verweis auf eine Tabelle im Text, das Wort Tabelle mit Tab. abgekürzt wird

% Laden der zweifarbigen Identbar (gegebenfalls auskommentieren)
%\input{parts/indentbar}

% Definition der Umgebung bibeinzug für manuelle Erstellung eines Eintrags des Literaturverzeichnisses
	\newenvironment{bibeinzug}{\hangindent=1cm}{\par}

% Definition des Befehls Bibzahl für manuelle Erstellung eines nummerierten Eintrags im Literaturverzeichnis
	\newcommand{\bibzahl}[2] { %
		\begin{tabular}{p{0.75cm}p{\textwidth-1.5cm}}
			[#1] &#2
		\end{tabular}
		}

%Definitionen der Abkürzungen und Symbole einfügen
%Ein extra Verzeichnis für Symbole erstellen
\newglossary[slg]{symbolslist}{syi}{syg}{Symbolverzeichnis}

%Den Punkt am Ende jeder Beschreibung deaktivieren
\renewcommand*{\glspostdescription}{}

%Glossar-Befehle anschalten
\makeglossaries

%				    	    %
%	Symbole definieren	%
%					    %
\newglossaryentry{symb:epsilon}{
	name=$\varepsilon$,
	description={Dehnung},
	sort=epsilon, type=symbolslist
}
\newglossaryentry{symb:phi}{
	name=$\varphi$,
	description={Winkel},
	sort=phi, type=symbolslist
}
\newglossaryentry{symb:E}{
	name=E,
	description={Elastizitätsmodul},
	sort=E, type=symbolslist
}
\newglossaryentry{symb:T}{
	name=T,
	description={Temperatur},
	sort=T, 
	type=symbolslist,
	symbol=\si{\kelvin}
}
\newglossaryentry{symb:sigma}{
	name=$\sigma$,
	description={Spannung},
	sort=sigma, 
	type=symbolslist,
	symbol=\si{\newton\per\millimetre\squared}
}
%							%
%	Abkürzungen definieren		%
%							%
%Befehls-Schema:
%\newacronym{label}{Abkürzung}{Bedeutung}
\newacronym{spz}{SPZ}{Sprachenzentrum}
\newacronym{pmv}{PMV}{Fachgebiet Papierfabrikation und Mechanische Verfahrenstechnik}




\newcommand{\TUtitle}[1]{\title{#1}}
\newcommand{\TUauthor}[1]{\author{#1}}
\newcommand{\TUdate}[1]{\date{#1}}

\newcommand{\TUchapter}[1]{\chapter{#1}}
\newcommand{\TUsection}[1]{\section{#1}}
\newcommand{\TUsubsection}[1]{\subsection{#1}}
\newcommand{\TUsubsubsection}[1]{\subsubsection{#1}}

\begin{document}
% Einfügen der Datei in der Aussehen und Inhalt der Titelseite hinterlegt ist
% Erklärungen zur korrekten Verwendung der Befehle finden sich in der TUD-Design Dokumentation
\TUtitle{\LaTeX-Vorlage für Abschlussarbeiten}
\TUauthor{\textbf{Skript zur Vorlesung \hfill Fachbereich Maschinenbau}}
\TUdate{\today}
%\sponsor{Sponsorenleiste}
%\setinstitutionlogo{}
%\settitlepicture{}
%\printpicturesize

% HINWEIS: Mit dem hyperref-Paket lassen sich entsprechende Informationen auch als Information in der PDF-Datei hinterlegen

%Römische Seitenzahlen
\pagenumbering{Roman}

%Titelseite erstellen
\pdfbookmark[0]{Titelseite}{titelseite}
\maketitle

%Kontaktseite einfügen und PDF-Lesezeichen setzen
\pdfbookmark[0]{Impressum}{impressum}
\begin{tabular}{c}\end{tabular}
\vfill
Donald Duck\\
Matrikelnummer: 1234567\\
Studiengang: B.Sc. Mechanical and Process Engineering
 
Master-/Bachelor-/Diplom-/Studienarbeit\\
Thema: Hier könnte Ihre Werbung stehen

Eingereicht: \today

Betreuer: Daniel Düsentrieb

Prof. Dr. Dagobert Duck\\
Fachgebiet Mustergebiet\\
Fachbereich Maschinenbau\\
Technische Universität Darmstadt\\
Hochschulstraße 1\\
64289 Darmstadt

%Abstract einfügen
\begin{abstract}
	Informationen zu Inhalten der Zusammenfassung entnehmen Sie bitte Kapitel 6.1 des Skripts zur Veranstaltung \textit{Wissenschaftliches Arbeiten und Schreiben für Maschinenbau-Studierende}.
		
	\blindtext
	
	\blindtext
	
	\blindtext
\end{abstract}% TUD-Design Abstract/Zusammenfassung
\include{parts/abstractalternativ} % klassisches Abstract

%Inhaltsverzeichnis einfügen und PDF-Lesezeichen setzen
\pdfbookmark[0]{Inhaltsverzeichnis}{toc} % PDF-Lesezeichen für das Inhaltsverzeichnis setzen, da dies nicht automatisch erfolgt
\tableofcontents

%Abspeichern des römischen Seitenzählers in Variable "savecounter" und Umstellung auf arabische Seitenzahlen
\clearpage
\newcounter{savecounter}						%anlegen der Variable "savecounter" zur Speicherung der letzten römischen Seitenzahl
\setcounter{savecounter}{\value{page}} %speichern der letzten römischen Seitenzahl in angelegter Variable zur Fortsetzung der römischen Nummerierung nach Ende des eigentlichen Dokuments
\pagenumbering{arabic} %Aktivierung von arabischen Seitenzahlen (setzt Counter für die Seitenzahl zurück auf 1)
%Laden der Teildokumente aus dem Ordner parts
\TUchapter{Einrichtung und Erläuterungen}
Um das modifizierte TU-Design zu verwenden, müssen Sie zunächst das offizielle TU-Design installieren. Die dazugehörigen, betriebssystemabhängigen Installationsanleitungen finden Sie unten.\\
In Ihrem Dokument laden Sie dann das Paket \emph{tustyle}. Standardmäßig ist die Layoutfarbe rot, sie können diese jedoch nach Belieben ändern. Hierzu laden sie das Paket mit der Option Ihrer gewünschten Farbe. Sollte Ihre Farbe noch nicht in der Klasse vorgesehen sein, können Sie diese selbstständig hinzufügen. Orientieren Sie sich hierbei an den Zeilen 144 bis 150 in dem Paket. Um eine neue Farbe zu definieren, benötigen Sie lediglich ihren RGB-Code.\\ \\
Beispiel: \textbackslash usepackage[blue]\{tustyle\} lädt das Paket \emph{tustyle} mit der Farbe \emph{blau}

Sollten Sie die offiziellen TU-Klassen verwenden, können Sie die Layoutfarbe als Option der Dokumentenklasse übergeben. Die möglichen Farben entnehmen Sie bitte der Handbuch des Corporate Designs.

Beispiel: \textbackslash documentclass[accentcolor=tud9c]\{tudreport\} lädt die Dokumentenklasse tudreport mit dem Farbe 9c (weinrot, Farbe dieses Dokuments)
\TUsection{Einrichtung unter Windows}
\TUsubsection{Setup}
Zu Beginn wird für den Start folgendes Setup vorgeschlagen:
\begin{itemize}
\item \LaTeX-Distribution: \href{http://miktex.org/download}{MiKTeX} [Zugriff: 13.12.2014]
\item \LaTeX-Editor: \href{http://texstudio.sourceforge.net/}{TeXstudio} [Zugriff: 13.12.2014]
\item Literaturverwaltung (optional):
\href{http://www.ulb.tu-darmstadt.de/service/literaturverwaltung_start/endnote_ulb/endnote.de.jsp}{Endnote}/\href{http://www.ulb.tu-darmstadt.de/service/literaturverwaltung_start/citavi_ulb/citavi_ulb.de.jsp}{Citavi} [Zugriff: 13.12.2014] mit TU-Lizenz der ULB oder das kostenfreie \href{http://jabref.sourceforge.net/download.php}{JabRef} [Zugriff: 13.12.2014]
\item PDF-Reader (optional): \href{http://blog.kowalczyk.info/software/sumatrapdf/download-free-pdf-viewer-de.html}{Sumatra PDF} [Zugriff: 13.12.2014] (Adobe Reader verhindert den Kompiliervorgang bei geöffnetem PDF-Dokument)
\end{itemize}
Es gibt eine Vielzahl kostenloser und kostenpflichtiger \LaTeX\ Distributionen (\href{http://www.tug.org/interest.html#free}{Auswahl} [Zugriff: 13.12.2014]) und Editoren (\href{http://en.wikipedia.org/wiki/Comparison_of_TeX_editors}{Übersicht} [Zugriff: 13.12.2014]) mit unterschiedlichem Funktionsumfang, mit denen persönliche Vorlieben erfüllt werden können.
Aus Gründen der Einfachheit und Reproduzierbarkeit beziehen sich Hilfestellungen sowie Tipps und Tricks dieser Einrichtungshilfe allerdings auf oben genanntes Setup. Die Installation der genannten Komponenten ist unproblematisch und sollte mit den jeweiligen Installationsprogrammen durchgeführt werden können.

Für eine problemlose Kompilierung unter Windows 7 wird empfohlen, die Miktexversion für 32 Bit zu verwenden.

\TUsubsection{Installation der TU-Design-Vorlage für \LaTeX}
Die Vorlage für Abschlussarbeiten greift für die Umsetzung des Corporate Designs der TU Darmstadt auf die \href{http://exp1.fkp.physik.tu-darmstadt.de/tuddesign/}{TUD-Design \LaTeX\ Vorlage} [Zugriff: 13.12.2014] zurück, welche von der Stabstelle Kommunikation und Medien genehmigt wurde. Diese hält die Vorgaben des Corporate Design Handbuchs (CDH) recht strikt ein (strikter als viele Fachgebiete dies bei den jeweils eigenen Word-Vorlagen tun), weshalb manche Anpassungen an Institutsvorgaben u.U. nur schwer umsetzbar sind, da sie gegen das CDH verstoßen.\\
Die notwendigen Pakete für die Verwendung der Vorlage für Abschlussarbeiten sind
\begin{itemize}
	\item das \href{http://exp1.fkp.physik.tu-darmstadt.de/tuddesign/latex/latex-tuddesign/latex-tuddesign_0.0.20100410.zip}{TUD-Design} [Zugriff: 13.12.2014]
	\item die \href{http://exp1.fkp.physik.tu-darmstadt.de/tuddesign/latex/tudfonts-tex/tudfonts-tex_0.0.20090806.zip}{TUD- Fonts} [Zugriff: 13.12.2014]
\end{itemize}
Hinweis: die TUD-Design Thesis Klasse, welche ebenfalls zum Download bereit steht, wird nicht benötigt.

\TUsubsubsection{Installation}
Für die Installation der TUD-Design Vorlage unter der MiKTeX Distribution unter Windows 7 kann folgende überarbeitete Anleitung verwendet werden. Sie basiert auf der \href{http://exp1.fkp.physik.tu-darmstadt.de/tuddesign/Win7_miktex29.html}{Installationsanleitung auf den Seiten der TUD-Design Vorlage} [Zugriff: 13.12.2014].

Hinweis: Für die Installation werden Administrator-Rechte benötigt
\begin{enumerate}
	\item\label{Schritt1} Entpacken der beiden Zip-Dateien (fonts und tuddesign) und anschließend aus den beiden Ordnern einen machen (ineinander kopieren und Verzeichnisse überschreiben)
	\item Öffnen der Eingabeaufforderung mit Administratorrechten: Start >> Programme >> Zubehör, dann Rechtsklick auf Eingabeaufforderung >> „Als Administrator ausführen“
	\item Mit \verb|cd <Pfad>| in das Verzeichnis wechseln, in dem der in \ref{Schritt1} angelegte Ordner \verb|texmf| liegt. Falls der texmf-Ordner auf einem anderen Laufwerk als \verb C liegt, muss beim Verzeichniswechsel der Parameter \verb|/d| angegeben werden:\\
	\textbf{Beispiel}: \\
	texmf-Ordner liegt unter \verb|E:\Test|\\
	Befehl: \verb|cd /d E:\Test|
	\item Löschen des Ordners \verb|texmf\fonts\map\dvipdfm| inklusive seines Inhalts mit folgendem Befehl\\
	\verb|rmdir /Q /S "C:\Users\Benutzername\TU-Design\texmf\fonts\map\dvipdfm"|
	\item Kopieren der Unterverzeichnisse von texmf in den Ordner \verb|\%PROGRAMFILES%\tuddesign\| mit folgendem Befehl:\\
	\verb|xcopy texmf "%PROGRAMFILES%\tuddesign" /E /I|\\
	Falls der xcopy Befehl fehlschlägt, liegen keine Administratorrechte vor (keine Schreibrechte für Programme-Ordner)
	\item Dannach folgenden Befehl ausführen:\\
	\verb|mo_admin|\\
	Zum Reiter Roots wechseln, den \emph{add}-Knopf drücken und das Verzeichnis \verb|\%PROGRAMFILES%\tuddesign| auswählen. (Unterordner tuddesign im Standard-Programmverzeichnis der Windows-Partition - i.d.R. auf \verb|C:\|)\\ Dann auf \emph{OK} klicken.
	\item In der Konsole folgendes eingeben:\\
	\verb|initexmf --admin --update-fndb|
	\item Anschließend Folgendes eingeben\\
	\verb|initexmf --edit-config-file=updmap|
	\item Folgende Zeilen in die sich öffnende Datei einfügen und speichern:\\
	Map 5ch.map\\
	Map 5fp.map\\
	Map 5sf.map
	\item Abschließend diesen Befehl ausführen\\
	\verb|initexmf --mkmaps|
\end{enumerate}
\TUsection{Einrichtung unter Linux}
\TUsubsection{Setup}
Zunächst wird für den Start folgendes Setup vorgeschlagen:
\begin{itemize}
	\item \LaTeX-Distribution: Texlive
	\item \LaTeX-Editor: TeXstudio
	\item PDF-Betrachter: Evince
\end{itemize}
Diese drei Programme können via Aptitude per Terminal installiert werden.

\TUsubsection{Installation der TU-Design-Vorlage für \LaTeX}
Die Vorlage für Abschlussarbeiten greift für die Umsetzung des Corporate Designs der TU Darmstadt auf die \href{http://exp1.fkp.physik.tu-darmstadt.de/tuddesign/}{TUD-Design \LaTeX\ Vorlage} [Zugriff: 13.12.2014] zurück, welche von der Stabstelle Kommunikation und Medien genehmigt wurde. Diese hält die Vorgaben des Corporate Design Handbuchs (CDH) recht strikt ein (strikter als viele Fachgebiete dies bei den jeweils eigenen Word-Vorlagen tun), weshalb manche Anpassungen an Institutsvorgaben u.U. nur schwer umsetzbar sind, da sie gegen das CDH verstoßen.\\
Die notwendigen Pakete für die Verwendung der Vorlage für Abschlussarbeiten sind
\begin{itemize}
	\item das \href{http://exp1.fkp.physik.tu-darmstadt.de/tuddesign/latex/latex-tuddesign/latex-tuddesign_0.0.20100410.zip}{TUD-Design} [Zugriff: 13.12.2014]
	\item die \href{http://exp1.fkp.physik.tu-darmstadt.de/tuddesign/latex/tudfonts-tex/tudfonts-tex_0.0.20090806.zip}{TUD-Fonts} [Zugriff: 13.12.2014]
\end{itemize}
Hinweis: die TUD-Design Thesis Klasse, welche ebenfalls zum Download bereit steht, wird nicht benötigt.

\TUsubsection{Installation}
Für die Installation der TUD-Design-Vorlage unter der Texlive-Distribution kann folgende Anleitung verwendet wird. Sie basiert auf der \href{http://exp1.fkp.physik.tu-darmstadt.de/tuddesign/debian.html}{Installationsanleitung auf den Seiten der TUD-Design-Vorlage} [Zugriff: 13.12.2014]\\
Wichtig ist, dass die folgende Anleitung nur im TU-Netz funktioniert.
\begin{enumerate}
	\item Zuerst müssen im Ordner \verb|/etc/apt/| in der Datei \verb|sources.list| die folgenden beiden Zeilen eingefügt werden.\\
		\verb|deb http://exp1.fkp.physik.tu-darmstadt.de/tuddesign/ lenny tud-design|\\
		\verb|deb-src http://exp1.fkp.physik.tu-darmstadt.de/tuddesign/ lenny tud-design|
	\item Anschließend müssen die folgenden drei Befehle eingegeben werden\\
		\verb|apt-get update|\\
		\verb|ap-get install debian-tuddesign-keyring|\\
		\verb|apt-get update|
	\item Nun können die TUD-Design-Klassen sowie die TUD-Schriftarten installiert werden. Hierzu werden folgende Befehle benötigt:\\
	\verb|apt-get install latex-tuddesign|\\
	\verb|apt-get install t1-tudfonts tex-tudfonts ttf-tudfonts|
\end{enumerate}
\TUsection{Einrichtung unter Mac}
\TUsubsection{Setup}
Zu Beginn wird für den Start folgendes Setup vorgeschlagen:
\begin{itemize}
\item \LaTeX-Distribution: \href{http://www.macports.org/install.php}{MacPorts} [Zugriff: 16.12.2014]
\item \LaTeX-Editor: \href{http://texstudio.sourceforge.net/}{TeXstudio} [Zugriff: 13.12.2014]
\item Literaturverwaltung (optional):
\href{http://www.ulb.tu-darmstadt.de/service/literaturverwaltung_start/endnote_ulb/endnote.de.jsp}{Endnote} [Zugriff: 13.12.2014] mit TU-Lizenz der ULB oder das kostenfreie \href{http://jabref.sourceforge.net/download.php}{JabRef} [Zugriff: 13.12.2014]
\item PDF-Reader (optional): Vorschau (Standardprogramm)
\end{itemize}
Es gibt eine Vielzahl kostenloser und kostenpflichtiger \LaTeX\ Distributionen (\href{http://www.tug.org/interest.html#free}{Auswahl} [Zugriff: 13.12.2014]) und Editoren (\href{http://en.wikipedia.org/wiki/Comparison_of_TeX_editors}{Übersicht} [Zugriff: 13.12.2014]) mit unterschiedlichem Funktionsumfang, mit denen persönliche Vorlieben erfüllt werden können.
Aus Gründen der Einfachheit und Reproduzierbarkeit beziehen sich Hilfestellungen sowie Tipps und Tricks dieser Einrichtungshilfe allerdings auf oben genanntes Setup. Die Installation der genannten Komponenten ist unproblematisch und sollte mit den jeweiligen Installationsprogrammen durchgeführt werden können.

\TUsubsection{Installation der TU-Design-Vorlage für \LaTeX}
Die Vorlage für Abschlussarbeiten greift für die Umsetzung des Corporate Designs der TU Darmstadt auf die \href{http://exp1.fkp.physik.tu-darmstadt.de/tuddesign/}{TUD-Design \LaTeX\ Vorlage} [Zugriff: 13.12.2014] zurück, welche von der Stabstelle Kommunikation und Medien genehmigt wurde. Diese hält die Vorgaben des Corporate Design Handbuchs (CDH) recht strikt ein (strikter als viele Fachgebiete dies bei den jeweils eigenen Word-Vorlagen tun), weshalb manche Anpassungen an Institutsvorgaben u.U. nur schwer umsetzbar sind, da sie gegen das CDH verstoßen.\\
Die notwendigen Pakete für die Verwendung der Vorlage für Abschlussarbeiten sind
\begin{itemize}
	\item das \href{http://exp1.fkp.physik.tu-darmstadt.de/tuddesign/latex/latex-tuddesign/latex-tuddesign_0.0.20100410.zip}{TUD-Design} [Zugriff: 13.12.2014]
	\item die \href{http://exp1.fkp.physik.tu-darmstadt.de/tuddesign/latex/tudfonts-tex/tudfonts-tex_0.0.20090806.zip}{TUD- Fonts} [Zugriff: 13.12.2014]
\end{itemize}
Hinweis: die TUD-Design Thesis Klasse, welche ebenfalls zum Download bereit steht, wird nicht benötigt.

\TUsubsubsection{Installation}
Für die Installation der TUD-Design Vorlage unter der Texlive-Distribution unter Mac kann folgende überarbeitete Anleitung verwendet werden. Sie basiert auf der \href{http://exp1.fkp.physik.tu-darmstadt.de/tuddesign/Win7_miktex29.html}{Installationsanleitung auf den Seiten der TUD-Design Vorlage} [Zugriff: 13.12.2014].

\begin{enumerate}
	\item Laden Sie die beiden oben genannten zip-Archive herunter.
	\item Öffnen Se ein Terminal (zu finden im Ordner \verb|Prgramme| unter \verb|Terminal.app|)
	\item Geben Sie dort zunächst den Befehl\\
		\verb|cd Downloads/|\\
		ein, um in den Ordner \verb|Downloads| zu wechseln. Sollten Ihre Downloads an einem anderen Ort gespeichert werden, verschieben Sie sie in den Ordner \verb|Downloads|.
	\item Geben Sie nun die folgenden zwei Befehle ein, um die heruntergeladenen zip-Archive an den richtigen Ort zu entpacken\\
		\verb|unzip latex_tuddesign_current.zip -d ~/Library/|\\
		\verb|unzip tudfonts-tex_current.zip -d ~/Library/|
	\item Geben Sie nun den Befehl\\
		\verb|sudo mktexlsr|\\
		ein.
	\item Geben Sie anschließend die folgenden drei Befehle ein.\\
		\verb|sudo updmap-sys --enable Map 5ch.map|\\
		\verb|sudo updmap-sys --enable Map 5fp.map|\\
		\verb|sudo updmap-sys --enable Map 5sf.map|
\end{enumerate}
\TUsection{Einstieg in \LaTeX}
Für einen fundierten Einsteig gibt es eine Vielzahl an Lehrbüchern, welche u.A. auch in der ULB verfügbar sind. Je nach Lerntyp ist aber auch learning-by-doing sehr gut möglich. Im Folgenden werden einige Internet-Informationsquellen aufgelistet die einen guten Einstieg in die Arbeit mit \LaTeX\ ermöglichen und/oder ein gutes Nachschlagewerk darstellen:
\begin{itemize}
	\item \href{http://en.wikibooks.org/wiki/LaTeX}{\LaTeX\ Wikibook} [Zugriff: 13.12.2014]
	\item \href{http://www.fernuni-hagen.de/imperia/md/content/zmi_2010/a026_latex_einf.pdf}{Manuela Jürgens \& Thomas Feuerstack: \LaTeX\ - eine Einführung und ein bisschen mehr... } [Zugriff: 13.12.2014]
	\item \href{ftp://ftp.fernuni-hagen.de/pub/pdf/urz-broschueren/broschueren/a0279510.pdf}{Manuela Jürgens: \LaTeX\ - Fortgeschrittene Anwendungen} [Zugriff: 13.12.2014]
	\item \href{http://latex.tugraz.at/latex/tutorial}{\LaTeX-Tutorial der Universität Graz} [Zugriff: 13.12.2014]
	\item \href{http://www.gidf.de/}{Geheimtip} [Zugriff: 13.12.2014]
\end{itemize}
PDF-Versionen sind, sofern vorhanden, dem Paket beigefügt. Die vorhandene Vorlage setzt das Wissen über den Inhalt der Dokumentation der TUD-Design Vorlage voraus.

Bei Problemen gibt es eine Vielzahl an deutschsprachigen und englischsprachigen Foren, in denen man Hilfe finden kann. Vor Eröffnung eines Beitrags sollte allerdings die Suchfunktion bemüht werden und bei der Erläuterung des Problems ein \href{http://www.golatex.de/wiki/Minimalbeispiel}{Minimalbeispiel} [Zugriff: 13.12.2014] angegeben werden. Speziell bei Problemen mit der TUD-Design Vorlage ist außerdem das \href{http://tuddesign-latex.fs-etit.de/index.php}{LaTeX-Forum des neuen TUD Designs} [Zugriff: 18.04.2014] zu empfehlen.

\TUsubsection{Anzeige des kompilierten PDF-Dokuments}
TeXstudio verfügt über eine interne PDF-Anzeige, welche den aktuellen Stand des Dokuments nach jedem Kompiliervorgang anzeigt. Unter Umständen kann allerdings auch die Arbeit mit einem externen PDF-Reader sinnvoll sein. 

Hier verursacht Adobe Reader allerdings das Problem, dass bei geöffnetem PDF-Dokument der Kompiliervorgang nicht durchgeführt werden kann, da ein geöffnetes Dokument schreibgeschützt ist. Aus diesem und anderen Gründen eignet sich Sumatra PDF als Reader für die Arbeit mit \LaTeX. Das angezeigte PDF-Dokument wird bei Verwendung von Sumatra PDF nach jedem Kompiliervorgang automatisch aktualisiert. 

Weiterhin ist es mit diesem Reader möglich, durch Doppelklick im PDF an die jeweilige Zeile im \LaTeX-Code zu springen. Auch ein Springen aus dem Code an die jeweilige Stelle des PDFs ist möglich. Die notwendigen Einstellungen, um das Springen zwischen dem PDF-Dokument und dem dazugehörigen \LaTeX-Code zu ermöglichen, können \href{http://robjhyndman.com/hyndsight/texstudio-sumatrapdf/}{dieser Anleitung} [Zugriff: 13.12.2014] entnommen werden.

\TUsubsection{Literaturverwaltung und Literaturverzeichnis}
%Die Verwendung eines Literaturverwaltungsprogramms zum Schreiben von wissenschaftlichen Arbeiten ist immer lohnenswert. Auf den meisten Suchportalen hat sich mittlerweile etabliert, dass für vorhandene Quellen die Möglichkeit angeboten wird, für die gängigsten Literaturverwaltungsprogramme automatisch einen Eintrag zu erzeugen.
%Die Literatureinträge können entweder mittels eines externen Programms wie Citavi, Endnote oder Jabref verwaltet oder direkt in \LaTeX\ in einer separaten Datei mit der Endung \emph{.bib} angelegt werden.
%
%Externe Programme ermöglichen durch Integration in \LaTeX-Editoren (vergleichbar mit der Integration in Word) bzw. den Export der Datenbank in ein \LaTeX-kompatibles Format ein einfaches Zitieren und automatisches Anlegen von Literaturverzeichnissen.

\paragraph{BibTeX}\noindent\\
Das Standardformat zur Zitation und zur Erzeugung von Literaturverzeichnissen in \LaTeX\ ist BibTeX. Viele Literaturverwaltungsprogramme wie Citavi, Endnote und Jabref unterstützen den automatischen Export der in der Datenbank der Literaturverwaltungssoftware hinterlegten Informationen in einem zu BibTeX kompatiblen Format.
Alternativ können die BibTeX-Einträge auch direkt in \LaTeX\ in einer separaten Datei mit der Endung \emph{.bib} angelegt werden. Genauere Informationen hierzu finden Sie  \href{http://en.wikibooks.org/wiki/LaTeX/Bibliography_Management\#BibTeX}{hier} [Zugriff: 13.12.2014].

\paragraph{natbib}\noindent\\
Ein weitverbreitetes Paket zur Erweiterung des Funktionsumfangs von \LaTeX\ für Naturwissenschaftler stellt \href{http://ftp.gwdg.de/pub/ctan/macros/latex/contrib/natbib/natbib.pdf}{Natbib} [Zugriff: 13.12.2014] dar. Natbib ermöglicht die Verwendung zusätzlicher Zitierstile wie beispielsweise die \glqq Harvard\grqq-Zitierweise und weitere Bibliographiestile.

\paragraph{biblatex}\noindent\\
Mit biblatex existiert eine recht junge Neuimplementierung der bibliographischen Funktionen für \LaTeX, welche inoffiziell als Nachfolger von BibTeX betrachtet wird. biblatex bietet den Vorteil, dass sämtliche Funktionen zur Gestaltung von Zitierstilen und Bibliographiestilen durch Optionen zugänglich gemacht werden. Dadurch entfällt die nicht triviale Programmierung von Stilen in BibTeX zur Anpassung an die jeweiligen Vorgaben.

Weiterhin bietet biblatex in Verbindung mit dem Bibliographie-Prozessor biber volle UTF-8 Unterstützung. Dadurch lässt sich die komplizierte und fehleranfällige Darstellung von deutschen Umlauten und Sonderzeichen in URLs mittels spezieller Befehle vermeiden.

biblatex definiert außerdem einige zusätzliche Eintrags- und Feldtypen, so dass beispielsweise Internetquellen ohne Workarounds über andere Eintragstypen mit dem Eintragstyp \glqq online\grqq\ gut dargestellt werden können. Zusätzlich reduziert sich die Anzahl der Kompilierdurchläufe bis zum fertigen PDF-Dokument auf einen Durchlauf im Vergleich zu drei Durchläufen bei BibTeX. Neben den genannten Vorteilen besitzt biblatex auch vollständige Kompatibilität zu BibTeX, so dass BibTeX-Datensätze ohne Probleme mit biblatex verarbeitet werden können.

Aus oben genannten Gründen baut die vorliegende Vorlage auf biblatex in Verbindung mit biber auf.
Nichtsdestotrotz bleibt festzuhalten, dass die Anpassung von Zitier- und Bibliographiestilen auch unter biblatex nicht einfach ist und ausreichende \LaTeX-Kenntnisse voraussetzt. Mit Blick auf die Zeiteffizienz sollte deshalb bei sehr expliziten Vorgaben des jeweiligen Fachbereichs für die Abschlussarbeit auch die manuelle Erstellung des Literaturverzeichnisses in Erwägung gezogen werden, sofern nicht die Darstellung Mittels einem der Pakete mit einem der weiter verbreiteten, vorhandenen Stile verhandelt werden kann.

MikTeX liefert die notwendigen Pakete für biblatex und biber bereits von Haus aus mit, so dass diese nur noch wie gewöhnlich geladen werden müssen. Außerdem unterstützt TeXstudio die Verwendung von biblatex und biber und es Bedarf deshalb keiner gesonderten Einstellungen im Editor. Der Kompilationsvorgang von biber wird standardmäßig mit der Funktionstaste F11 aufgerufen. Der Standardkompiliervorgang PdfLaTeX (F1), der das PDF erzeugt, bindet die von biber erzeugten Dateien automatisch ein.

\TUsection{Symbol- und Abkürzungsverzeichnis}
Teil vieler naturwissenschaftlicher Arbeiten ist ein Symbol- und/oder Abkürzungsverzeichnis. \LaTeX\ bietet hierfür standardmäßig keine spezielle Lösung, so dass die Verzeichnisse manuell erstellt werden oder mittels Paketen wie beispielsweise \textit{glossaries} oder \textit{nomencl} realisiert werden müssen.

Auch hier bietet die manuelle Erstellung mehr Freiheiten und einen geringeren Einarbeitungs- und Einrichtungsaufwand für die reine Verzeichniserstellung, sofern nicht die Zusatzfunktionen der Pakete benötigt werden. Für die manuelle Erstellung mittels einer Tabelle bietet sich das Paket \emph{longtable} an.

Die Vorlage bietet jeweils ein einfaches Beispiel für die Einbindung mittels des Pakets \emph{glossaries}, sowie ein Beispiel mittels einer Tabelle. Für die Verwendung der glossaries-Variante wird \textit{makeindex} benötigt. In der TeXstudio-Konfiguration ist makeindex bereits hinterlegt (Shortcut F12), allerdings muss dort folgende Anpassung für die verwendete Variante vorgenommen werden:\newline
Unter \verb|Optionen| $\rightarrow$ \verb|TeXstudio konfigurieren| $\rightarrow$ \verb|Befehle| muss beim Eintrag \verb|Makeindex| folgende Befehlszeile stehen (ohne Absatz, alles in einer Zeile): 
\begin{verbatim}
makeindex -s %.ist -t %.alg -o %.acr %.acn | 
makeindex -s %.ist -t %.glg -o %.gls %.glo | 
makeindex -s %.ist -t %.slg -o %.syi %.syg
\end{verbatim}

Um die automatischen Verzeichnisse einzubinden, muss das Dokument zunächst normal kompiliert werden (entweder mit pdflatex oder mit dvilatex), anschließend werden die Befehle \emph{glossary} (in TeXstudio standardmäßig per F10) und \emph{Index} (in TeXstudio standardmäßig per F12) aufgerufen und im Anschluss wird das Dokument nochmals normal kompiliert (wieder mit pdflatex oder dvilatex).

\TUsection{Abschließende Bemerkung}
Auch wenn die manuelle Erstellung in den beiden vorherigen Kapiteln immer im Hinblick auf den häufig vorhandenen Zeitdruck bei einer Abschlussarbeit als zu berücksichtigende Alternative genannt wurde, bleibt dennoch zu sagen, dass einmal erlerntes Wissen über \LaTeX\ und dessen Erweiterungspakete sowie geschriebener Programmcode immer den Vorteil der einfachen Wiederverwendbarkeit für spätere Arbeiten bietet. Es bleibt also letztendlich immer eine Einzelfallentscheidung, ob der zeitliche Aufwand für die Einarbeitung in ein neues Paket o.Ä. (als eine u.U. auch längerfristig fruchtende Lösung) gerechtfertigt oder eine kurzfristig händische Lösung zu bevorzugen ist.
\TUchapter{Beispiele}
\TUsection{Bild}			
	\begin{figure}[h]
		\centering
		\includegraphics[scale=0.5]{wsmb}		
		\caption{Logo }
	\end{figure}

\TUsection{Tabelle}
\begin{table}[h]
	\centering
	\begin{tabular}{cccc}
		\toprule
		Buchstabe &Zahl &BUCHSTABE &Zahl Zahl \\\midrule
		a &1 &A &11 \\
		b &2 &B &22 \\\bottomrule
	\end{tabular}
	\caption{Zahlen und Buchstaben}
\end{table}

\TUsection{Text mit Zitat}
Die ist ein Beispieltext mit einer zitierten Quelle \cite{Blomeke.2006}. Und noch ein Zitat \cite{Hering.2007}, auf das ein weiteres Zitat aus einer Monographie folgt \cite[23]{Karmasin.2012}.

\TUsection{Zahlen und Einheiten}
Für eine einheitliche Darstellung von Zahlen und Einheiten wird das Paket \emph{siunitx} verwendet. Diese führt die Makros
\begin{itemize}
	\item \verb|\si| für Einheiten
		\begin{itemize}
			\item Beispiel:\\
			\verb|Ein \si{\newton} ist definiert als \si{\kilogram\meter\per\second\squared}.|\\
			führt zu\\
			Ein \si{\newton} ist definiert als \si{\kilogram\meter\per\second\squared}.
		\end{itemize}
	\item \verb|\SI| für Zahlen und Einheiten
		\begin{itemize}
			\item Beispiel: \verb|eine Spannung der Höhe \SI{2,386e3}{\newton\per\milli\meter\squared}|\\
			führt zu\\
			eine Spannung der Höhe \SI{2,386e3}{\newton\per\milli\meter\squared}
		\end{itemize}
	\item \verb|\SIlist| für Aufzählungen von Zahlen und Einheiten
		\begin{itemize}
			\item Beispiel: \verb|\SIlist{10;100;1000}{\kilogram}|\\
			führt zu\\
			\SIlist{10;100;1000}{\kilogram}
		\end{itemize}
	\item \verb|\SIrange| für die Darstellungen von Bereichen
		\begin{itemize}
			\item Beispiel: \verb|\SIrange{300}{500}{\kelvin}|\\
			führt zu\\
			\SIrange{300}{500}{\kelvin}
		\end{itemize}
\end{itemize}
ein. Die Darstellung von Zahlen und Einheiten können zentral in der Präambel des Dokuments oder als Parameter der Makros von Fall zu Fall neu definiert werden. So ist sowohl Konsistenz als auch Flexibilität gewährleistet.\\
Darüber hinaus gibt es noch viele weitere Anwendungsbereiche. Weitere Informationen zur Verwendung und Konfiguration sind der beiliegenden Dokumentation des Pakets zu entnehmen.
%Setzen des Seitenzählers auf den zuvor gespeicherten römischen Seitenzählers und Umstellung auf römische Seitenzahlen
\pagenumbering{Roman}					%Aktivierung römischer Seitenzahlen
\setcounter{page}{\value{savecounter}}		%Aufruf der zuvor gespeicherten zuletzt verwendeten römischen Seitenzahl
%Literaturverzeichnis einfügen
\printbibliography
%Ehrenwörtliche Erklärung einfügen (je nach dem eine Variante auskommentieren)
\include{parts/ehrenwortde}
\TUchapter{Declaration of Academic Integrity}
\vspace{11pt}
\paragraph{Thesis Statement pursuant to § 22 paragraph 7 of APB TU Darmstadt}\noindent\\
I herewith formally declare that I have written the submitted thesis independently. I did not use any outside support except for the quoted literature and other sources mentioned in the paper. I clearly marked and separately listed all of the literature and all of the other sources which I employed when producing this academic work, either literally or in content. This thesis has not been handed in or published before in the same or similar form.\newline
In the submitted thesis the written copies and the electronic version are identical in content.\vspace{40pt}

\noindent Date:\hspace{0.4\textwidth}Signature:
\vspace*{2cm}

%Abbildungsverzeichnis einfügen
\listoffigures

%Tabellenverzeichnis einfügen
\listoftables

% Abkürzungsverzeichnis einfügen
\newpage
\glsaddall
\setlength{\glsdescwidth}{14cm}
\printglossary[type=\acronymtype,style=long, title=Abkürzungsverzeichnis]
%Alternatives Abkürzungsverzeichnis einfügen
\include{parts/tababkuerzungsverzeichnis}
%Symbolverzeichnis einfügen
\glsaddall
\setlength{\glsdescwidth}{12.5cm}
\printglossary[type=symbolslist,style=longheader, title=Symbolverzeichnis]
%Alternatives Symbolverzeichnis einfügen
\include{parts/tabsymbolverzeichnis}
%Anhang einfügen
\appendix
\chapter{Anhang}
\section{Ein Anhang}
Hier gibt es etwas zu sagen oder auch nicht.
\subsection{Teil eines Anhangs}
\blindtext
\subsection{Noch ein Teil eines Anhangs}
\blinditemize
\section{Noch ein Anhang}
Hier gibt es etwas zu sagen oder auch nicht.
\subsection{Teil des weiteren Anhangs}
\blindenumerate
\subsection{Noch ein Teil des weiteren Anhangs}
%	\blindmathpaper

\blindtext
\begin{align}
\bar x=\frac 1n\cdot\sum_{i=1}^{i=n}x_i=\frac{x_1+x_2+\ldots+x_n}{n}
\end{align}
\blindtext
\begin{align}
\int_0^\infty e^\mathrm{-ax^2}\mathrm dx=\frac 12\sqrt{\int_{-\infty}^\infty e^\mathrm{-ax^2}}\mathrm dx\cdot\int_{-\infty}^\infty e^\mathrm{-ay^2}\mathrm dy=\frac 12\sqrt{\frac{\pi}{\alpha}}
\end{align}
\blindtext
\begin{align}
\sum_{k=0}^\infty a_0g^k=\lim_{n\to\infty}\sum_{k=0}^na_0q^k=\lim_{n\to\infty}a_0\frac{1-q^{n+1}}{1-q}=\frac{a_0}{1-q}
\end{align}
\blindtext
\begin{align}
x_{1,2}=\frac{-b\pm\sqrt{b^2-4ac}}{2a}=\frac{-p\pm\sqrt{p^2-4q}}{2}
\end{align}
\blindtext
\begin{align}
\frac{\partial^2\Phi}{\partial x^2}+\frac{\partial^2\Phi}{\partial y^2}+\frac{\partial^2\Phi}{\partial z^2}=\frac{1}{c^2}\frac{\partial^2\Phi}{\partial t^2}
\end{align}
\blindtext
\end{document}